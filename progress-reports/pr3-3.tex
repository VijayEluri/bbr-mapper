% Template for progress reports
% For examples on how to use latex see, for example,
% http://www.mat.bham.ac.uk/R.W.Kaye/latex/examples.htm
%
% Andy Sayler
% Constantin Berzan

\documentclass[12pt]{article}

\usepackage{times}
\usepackage{latexsym}
\usepackage{xspace}
\usepackage{times}
\usepackage{epsfig}
\usepackage{hyperref}
\usepackage{url}

\setlength {\topmargin} {0 mm}
\setlength {\headsep} {0 mm}
\setlength {\headheight} {0 in}
\setlength {\voffset} {0 mm}
\setlength {\oddsidemargin} {0 mm}
\setlength {\evensidemargin} {0 mm}
\setlength {\hoffset} {0 mm}
\setlength {\textwidth} {6.5 in}
\setlength {\textheight} {9 in}

\hypersetup{
    colorlinks=true
}

\begin{document}

\title{BBR Progress Report 003:\\ Autonomous 2-D Mapping of a Building Floor}
\author{Andy Sayler \& Constantin Berzan}
\maketitle

\begin{abstract}
This week we succeeded in obtaining odometry data from the Videre base. We
performed initial odometry testing, analyzed the data, and decided on what
further tests to run. We came up with a high-level architecture for our SLAM
system, as well as a testing plan in the ADE simulator.
\end{abstract}

\emph{The code for this project is located at
\url{http://github.com/cberzan/bbr-mapper}}

\section{Introduction}
SLAM is the problem of Simultaneous Localization and Mapping using a mobile
robot.  We aim to develop a proof-of-concept SLAM system adhering to the
behavior-based philosophy.  The robot will use odometry and laser range data to
navigate the first floor of Halligan, and produce an image representation of
the floor map.


\section{Project Summary}

We will be developing our robot using a schema (or possibly hybrid)
architecture. The robot will utilize schema based functions to manifest the
following behaviors:

\begin{itemize}
    \setlength{\itemsep}{0pt}
    \setlength{\parskip}{0pt}
    \setlength{\parsep}{0pt}
    \item Avoid obstacles
    \item Avoid local minima
    \item Seek new areas
    \item Utilize SLAM (laser + odometry) to deduce current location
    \item Utilize SLAM (laser + odometry) to generate persistent environmental
          map
\end{itemize}

We will utilize the ADE robotics environment to complete our implementation.
The first objective will be to create code capable of navigating our robot
through unfamiliar environments and exploring these environments to their full
potential without getting stuck or colliding with obstacles. 

The next goal will be to build a map of the environment and provide
localization abilities. This will be done by implementing a basic SLAM
system using laser and odomtery data.  We aim to develop a 2D floor plan map
using data from our SLAM system.\\

Should we complete the initial scope of this project ahead of schedule, we may
opt to pursue one or more of the following extensions:

\begin{itemize}
    \setlength{\itemsep}{0pt}
    \setlength{\parskip}{0pt}
    \setlength{\parsep}{0pt}
    \item Utilize vision data in SLAM system
    \item Utilize vision data in 2D map generation
    \item Compare performance of our SLAM system to Carmen
    \item Augment SLAM system with additional sensor packages (radio ranging,
          etc)
\end{itemize}


\section{Problems tackled}

Literature review:
\begin{itemize}
    \setlength{\itemsep}{0pt}
    \setlength{\parskip}{0pt}
    \setlength{\parsep}{0pt}
    \item SLAM for Dummies tutorial -- both
    \item Carmen -- Andy
    \item Vision-based SLAM -- Constantin
\end{itemize}

\noindent{Discussed and refined project scope:}
\begin{itemize}
    \setlength{\itemsep}{0pt}
    \setlength{\parskip}{0pt}
    \setlength{\parsep}{0pt}
    \item Opted to implement SLAM ourselves
    \item Decided to use SLAM data as primary source for persistent 2D map data
\end{itemize}

\noindent{Logistical issues:}
\begin{itemize}
    \setlength{\itemsep}{0pt}
    \setlength{\parskip}{0pt}
    \setlength{\parsep}{0pt}
    \item Decided on workflow and code layout for interfacing our code with ADE
    \item Figured out bootstrapping of ADE registry and simulator
    \item Wrote SSH scripts for easy remote operation of the robot
\end{itemize}

\noindent{Discussed Odometry Testing Setup:}
\begin{itemize}
    \setlength{\itemsep}{0pt}
    \setlength{\parskip}{0pt}
    \setlength{\parsep}{0pt}
    \item Linear Distance Testing (i.e. Drive 10m, then measure actual
      distance and angle)
    \item Start/Stop Testing (i.e. drive 1m in 10cm steps issuing
      start and stop commands between each step to measure start/stop drift)
    \item Rotational Distance Testing (i.e. Turn 180 degrees, then measure
      actual angle.)
    \item Acceleration Test (i.e. drive 10m and use internal timer to log
      distance at each tenth second interval to compute robot dynamics)
\end{itemize}

\noindent{Created Map of Halligan for the Simulator:}
\begin{itemize}
    \setlength{\itemsep}{0pt}
    \setlength{\parskip}{0pt}
    \setlength{\parsep}{0pt}
    \item The map was auto-generated by extracting all rectangles from a SVG
          image, which was drawn by hand
    \item The svg-to-xml conversion script is reusable, and allows the map to be
          modified easily, using a graphical editor such as Inkscape
    \item The goal is to capture the complexity of the environment, while not
          necessarily drawing it to scale
\end{itemize}

\noindent{Odometry Testing:}
\begin{itemize}
    \setlength{\itemsep}{0pt}
    \setlength{\parskip}{0pt}
    \setlength{\parsep}{0pt}
    \item Performed drive-forward test for 1 and 2 meters
    \item Ascertained need for internal odometry resets between trials, and
          scheduled more testing.
    \item Studied odometry data to understand the coordinate system of the
          robot.
\end{itemize}

\noindent{High-level Design:}
\begin{itemize}
    \setlength{\itemsep}{0pt}
    \setlength{\parskip}{0pt}
    \setlength{\parsep}{0pt}
    \item Discussed high-level design of the SLAM system. We will have our
          Extended Kalman Filter in its own ADE server. We will also have a
          separate ADE server for landmark extraction (from laser data). Our
          robot will be ran by a third ADE server, modeled after ArchImpl.
\end{itemize}

\noindent{Implementation Plan:}
\begin{itemize}
    \setlength{\itemsep}{0pt}
    \setlength{\parskip}{0pt}
    \setlength{\parsep}{0pt}
    \item The initial implementation is going to work in the ADE simulator,
          using beacons as landmarks. This will allow us to implement and test
          the EKF separately from the landmark detection algorithm.
    \item The next step will be extracting landmarks from laser data. This can
          also be tested in simulation.
    \item Output useful mapping data. Test it in the simulator.
    \item Test everything in the real world. Tweak Kalman filter.
\end{itemize}



\section{Next Steps}

\begin{itemize}
    \setlength{\itemsep}{0pt}
    \setlength{\parskip}{0pt}
    \setlength{\parsep}{0pt}
    \item More odometry testing, with proper resetting -- both
    \item Figure out how to use odometry data in ADESim -- Constantin
    \item Find published information about errors of the provided LRF -- Andy
    \item Begin coding of the Extended Kalman Filter -- both
\end{itemize}

\end{document}
