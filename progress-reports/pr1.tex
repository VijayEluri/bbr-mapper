% Template for progress reports
% For examples on how to use latex see, for example,
% http://www.mat.bham.ac.uk/R.W.Kaye/latex/examples.htm

\documentclass[12pt]{article}
\usepackage{times}
\usepackage{latexsym}
\usepackage{xspace}
\usepackage{times}
\usepackage{epsfig}
\setlength {\topmargin} {0 mm}
\setlength {\headsep} {0 mm}
\setlength {\headheight} {0 in}
\setlength {\voffset} {0 mm}
\setlength {\oddsidemargin} {0 mm}
\setlength {\evensidemargin} {0 mm}
\setlength {\hoffset} {0 mm}
\setlength {\textwidth} {6.5 in}
\setlength {\textheight} {9 in}

\begin{document}

\title{BBR Progress Report NNN:\\YOUR TITLE}

\author{YOUR NAME}

\maketitle

\begin{abstract}
Summarize here what the progress report contains in one or two sentences.
\end{abstract}


\section{Introduction}

Work on the introduction--since the introduction can be the same for
all group members except for the outline of the rest of the paper
(which contains what each member specifically did), this can be done
early on and can be a group effort.  Hence, try to write it up already
for an early progress report, if possible in its final form.


\section{Project Summary}

Summarize your project here--you can take your summary from the final
project proposals (this should be self-contained, so that somebody who
reads it understands immediately what the project is about and how it
will be accomplished).

\section{Problems tackled}

Describe the problems that have tackled since the last progress
report.  In particular, itemize the accomplished tasks, report on what
each group member did, what problems were encountered, and how they
were solved

\section{Next Steps}

Summarize the state of the project and lay out the next steps (until
the next progress report).  In particular, itemize the next steps and
assigned names to duties.

\end{document}
