% Template for progress reports
% For examples on how to use latex see, for example,
% http://www.mat.bham.ac.uk/R.W.Kaye/latex/examples.htm
%
% Andy Sayler
% Constantin Berzan

\documentclass[12pt]{article}
\usepackage{times}
\usepackage{latexsym}
\usepackage{xspace}
\usepackage{times}
\usepackage{epsfig}
\setlength {\topmargin} {0 mm}
\setlength {\headsep} {0 mm}
\setlength {\headheight} {0 in}
\setlength {\voffset} {0 mm}
\setlength {\oddsidemargin} {0 mm}
\setlength {\evensidemargin} {0 mm}
\setlength {\hoffset} {0 mm}
\setlength {\textwidth} {6.5 in}
\setlength {\textheight} {9 in}

\begin{document}

\title{BBR Progress Report 001:\\ Autonomous 2-D Mapping of a Building Floor}
\author{Andy Sayler \& Constantin Berzan}
\maketitle

\begin{abstract}
This week we completed an initial review of the SLAM literature and refined the
scope of our project.
\end{abstract}


\section{Introduction}
SLAM is the problem of Simultaneous Localization and Mapping using a mobile
robot.  We aim to develop a proof-of-concept SLAM system adhering to the
behavior-based philosophy.  The robot will use odometry and laser range data to
navigate the first floor of Halligan, and produce an image representation of the
floor map.


\section{Project Summary}

We will be developing our robot using a schema (or possibly hybrid)
architecture. The robot will utilize schema based functions to manifest the
following behaviors:

\begin{itemize}
    \setlength{\itemsep}{0pt}
    \setlength{\parskip}{0pt}
    \setlength{\parsep}{0pt}
    \item Avoid obstacles
    \item Avoid local minima
    \item Seek new areas
    \item Utilize SLAM (laser + odometry) to deduce current location
    \item Utilize SLAM (laser + odometry) to generate persistent environmental
          map
\end{itemize}

We will utilize the ADE robotics environment to complete our implementation.
The first objective will be to create code capable of navigating our robot
through unfamiliar environments and exploring these environments to their full
potential without getting stuck or colliding with obstacles. 

The next goal will be to build a map of the environment and provide localization
abilities. This will be done by implementing a basic SLAM system using laser and
odomtery data.  We aim to develop a 2D floor plan map using data from our SLAM
system.\\

Should we complete the initial scope of this project ahead of schedule, we may
opt to pursue one or more of the following extensions:

\begin{itemize}
    \setlength{\itemsep}{0pt}
    \setlength{\parskip}{0pt}
    \setlength{\parsep}{0pt}
    \item Utilize vision data in SLAM system
    \item Utilize vision data in 2D map generation
    \item Compare performance of our SLAM system to Carmen
    \item Augment SLAM system with additional sensor packages (radio ranging,
          etc)
\end{itemize}


\section{Problems tackled}

Literature review:
\begin{itemize}
    \setlength{\itemsep}{0pt}
    \setlength{\parskip}{0pt}
    \setlength{\parsep}{0pt}
    \item SLAM for Dummies tutorial -- both
    \item Carmen -- Andy
    \item Vision-based SLAM -- Constantin
\end{itemize}

\noindent{Discussed and refined project scope:}
\begin{itemize}
    \setlength{\itemsep}{0pt}
    \setlength{\parskip}{0pt}
    \setlength{\parsep}{0pt}
    \item Opted to implement SLAM ourselves
    \item Decided to use SLAM data as primary source for persistent 2D map data
\end{itemize}


\section{Next Steps}

\begin{itemize}
    \setlength{\itemsep}{0pt}
    \setlength{\parskip}{0pt}
    \setlength{\parsep}{0pt}
    \item Review ADE architecture for optimal design and workflow -- both
    \item Build simple map of Halligan for the simulator -- Constantin
    \item Determine metrics of Videre base (laser range, odometry accuracy, etc)
          -- Andy
    \item Review Kalman filtering technique -- both
\end{itemize}

\end{document}
